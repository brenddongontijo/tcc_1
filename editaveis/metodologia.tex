\chapter{Metodologia}

\section{Métodos Ágeis}
Os negócios atualmente operam em um ambiente global sujeito a rápidas mudanças sendo crucial
estar preparado para novas oportunidades de mercado, mudanças de condições econômicas e ao surgimento
de produtos e serviços concorrentes. O \textit{software} é um dos componentes cruciais para a realização
de todas as operações de negócio e realizá-lo de forma rápida também é uma maneira de aproveitar novas
oportunidades e responder às pressões competitivas. \cite{sommerville_2006}

Segundo \cite{sommerville_2006}, processos de desenvolvimento ágeis geralmente são iterativos onde a
especificação, projeto, desenvolvimento e teste são intercalados. O \textit{software} é desenvolvido
em uma série de incrementos e cada incremento fornece uma nova funcionalidade ao sistema. As duas principais
vantagens de se adotar uma abordagem incremental para o desenvolvimento de \textit{software} são:

\begin{itemize}
    \item Entrega acelerada dos serviços ao cliente. Os clientes poderão obter valor do sistema com incrementos iniciais.
    \item Engajamento do usuário com o sistema. Os usuários do sistema devem estar envolvidos no processo
    de desenvolvimento dando \textit{feedback} à equipe de desenvolvimento sobre os incrementos entregues.
\end{itemize}

\section{\textit{Extreme Programming}}
Segundo \cite{beck_2004}, \textit{Extreme Programming} (XP) é uma filosofia de desenvolvimento de
\textit{software} baseada nos valores de comunicação, \textit{feedback}, simplicidade, coragem e respeito.
Por ser um tipo de metodologia ágil de desenvolvimento de \textit{sofwtare} são defendidas \textit{releases}
frequentes, as quais devem possuir curtos ciclos de desenvolvimento, destinadas à aumentar a produtividade, qualidade de \textit{software} e introduzir pontos de controle para que novas necessidades dos clientes sejam atendidas.

Práticas do XP:

\begin{itemize}
    \item Testar cedo, com frequência e automação.
    \item \textit{Design} incremental.
    \item Implantação diária.
    \item Envolvimento do cliente.
    \item Integração contínua.
    \item Ciclos de desenvolvimento curtos.
    \item Planejamento incremental.
    \item Programação em pares.
\end{itemize}

\section{Scrum}
Scrum.
O que é sprint.
Reunião de planejamento de sprint com orientador e cliente.

\section{Sofware Livre}

\section{Licença de Software}
MIT

\section{Cronograma}
