\chapter{Métodos Empíricos em Engenharia de \textit{Software}}

\section{Métodos Ágeis}
Os negócios atualmente operam em um ambiente global sujeito a rápidas mudanças sendo crucial
estar preparado para novas oportunidades de mercado, mudanças de condições econômicas e ao surgimento
de produtos e serviços concorrentes. O \textit{software} é um dos componentes cruciais para a realização
de todas as operações de negócio e realizá-lo de forma rápida também é uma maneira de aproveitar novas
oportunidades e responder às pressões competitivas \cite{sommerville_2006}.

O manifesto ágil \cite{beck2001agile} consiste na base que fundamenta o desenvolvimento ágil de
\textit{software} sendo composto de quatro valores fundamentais:

\begin{itemize}
    \item \textbf{Os indivíduos e suas interações} acima de procedimentos e ferramentas;
    \item \textbf{O funcionamento do \textit{software}} acima de documentação abrangente;
    \item \textbf{A colaboração dos clientes} acima da negociação de contratos;
    \item \textbf{A capacidade de resposta a mudanças} acima de um plano pré-estabelecido.
\end{itemize}

``Enquanto há valor nos itens à direita, valorizamos mais os itens à esquerda'' \cite{beck2001agile}.

Além desses valores são definidos 12 princípios de agilidade:

\begin{itemize}
    \item Nossa maior prioridade é satisfazer o cliente, através da entrega adiantada e contínua de \textit{software} de valor;
    \item Aceitar mudanças de requisitos, mesmo no fim do desenvolvimento. Processos ágeis se adequam a mudanças, para que o cliente possa tirar vantagens competitivas;
    \item Entregar \textit{software} funcionando com freqüencia, na escala de semanas até meses, com preferência aos períodos mais curtos;
    \item Pessoas relacionadas à negócios e desenvolvedores devem trabalhar em conjunto e diariamente, durante todo o curso do projeto;
    \item Construir projetos ao redor de indivíduos motivados. Dando a eles o ambiente e suporte necessário, e confiar que farão seu trabalho;
    \item O Método mais eficiente e eficaz de transmitir informações para, e por dentro de um time de desenvolvimento, é através de uma conversa cara a cara;
    \item \textit{Software} funcional é a medida primária de progresso;
    \item Processos ágeis promovem um ambiente sustentável. Os patrocinadores, desenvolvedores e usuários, devem ser capazes de manter indefinidamente, passos constantes;
    \item Contínua atenção à excelência técnica e bom design, aumenta a agilidade;
    \item Simplicidade: a arte de maximizar a quantidade de trabalho que não precisou ser feito;
    \item As melhores arquiteturas, requisitos e designs emergem de times auto-organizáveis;
    \item Em intervalos regulares, o time reflete em como ficar mais efetivo, então, se ajustam e otimizam seu comportamento de acordo.
\end{itemize}

Nem todos os processos ágeis aplicam tais princípios de maneira igualitária e alguns modelos escolhem ignorar (ou ao menos minimizar) a importância de um ou mais princípios \cite{pressman_2009}.

Processos de desenvolvimento ágeis geralmente são iterativos onde a
especificação, projeto, desenvolvimento e teste são intercalados. O \textit{software} é desenvolvido
em uma série de incrementos e cada incremento fornece uma nova funcionalidade ao sistema \cite{sommerville_2006}. As duas principais
vantagens de se adotar uma abordagem incremental para o desenvolvimento de \textit{software} são:

\begin{itemize}
    \item Entrega acelerada dos serviços ao cliente. Os clientes poderão obter valor do sistema com incrementos iniciais;
    \item Engajamento do usuário com o sistema. Os usuários do sistema devem estar envolvidos no processo
    de desenvolvimento dando \textit{feedback} à equipe de desenvolvimento sobre os incrementos entregues.
\end{itemize}

    \subsection{\textit{Extreme Programming}}
    \textit{Extreme Programming} (XP) é uma filosofia de desenvolvimento de \textit{software} baseada nos valores de comunicação, simplicidade, \textit{feedback} e coragem. É designado para se trabalhar com projetos que podem ser
    construídos por equipes de dois a dez programadores, os quais não devem possuir limitações
    por causa do ambiente computacional e possam ser capazes de realizar testes de
    \textit{software} em uma fração de um dia \cite{beck_2004}.

    No contexto deste trabalho, o projeto foi desenvolvido pelo próprio autor utilizando os valores e princípios do XP. As práticas utilizadas para o desenvolvimento do sistema foram:

    \begin{itemize}
        \item Teste: escrever testes unitários, os quais devem funcionar de maneira adequada para que
        o desenvolvimento do sistema continue. A cada término de uma funcionalidade eram realizados testes
        seus testes, objetivando uma cobertura de no mínimo 90\%.
        \item Refatoração: reestruturação do sistema sem alteração de seu comportamento, objetivando
        remover duplicidade, simplificação, aumento de flexibilidade e melhoramento da comunicação. As propostas de refatorações eram identificadas nas reuniões com o orientador
        e geralmente realizadas em iterações subsequentes.
        \item Padrões de Codificação: escrever o código de acordo com boas práticas de programação, enfatizando uma comunicação através do código. Pelo código ser escrito na linguagem \textit{python}
        procurou-se sempre utilizar as convenções definidas pela PEP 8\footnote{\url{https://www.python.org/dev/peps/pep-0008/}}.
        \item Integração Contínua: integrar e construir o sistema muitas vezes ao dia, sempre que uma tarefa
        for concluída. A Integração Contínua foi realizada com auxílio da ferramenta Gitlab CI, onde para cada novo \textit{commit} no repositório é gerada uma \textit{build} verificando se os testes e as normas do PEP 8 foram realizados de forma correta.
    \end{itemize}

    \subsection{Scrum}
    Scrum é um \textit{framework} estrutural flexível utilizado
    para gerenciar o desenvolvimento de produtos complexos empregando, com o auxílio de \textit{sprints}, uma abordagem iterativa
    e incremental para aperfeiçoar a previsibilidade e o controle de riscos. Diferentes papéis são definidos pelo Scrum: Product Owner, Time de Desenvolvimento e Scrum Master \cite{scrum_guide}.

    O \textit{Product Owner} (PO), ou dono do produto, é o responsável por maximizar o valor
    do produto e do trabalho do Time de Desenvolvimento realizando o gerenciamento
    do \textit{Backlog} do Produto.

    O Time de Desenvolvimento é responsável por realizar o desenvolvimento de \textit{software} e entregar um incremento de \textit{software} ao final de cada \textit{sprint}.

    \textit{Sprint} é um \textit{time-boxed} de um mês ou menos, durante o qual
    uma versão incremental potencialmente utilizável do produto é criado.
    Uma nova Sprint inicia imediatamente após a conclusão da Sprint anterior.

    O Scrum Master é responsável por garantir que o Scrum seja entendido e aplicado, ou seja, que o Time Scrum siga à teoria, práticas e regras do Scrum.

    A tabela \ref{equipe_scrum}, apresenta como foram definidos os papéis no contexto do projeto.

    \begin{table}[!htbp]
        \centering
        \caption{Equipe Scrum. Fonte: autor}
        \label{equipe_scrum}
        \begin{tabular}{|p{5cm}|p{4cm}|}
        \hline
        \textbf{Membro}                                                           & \textbf{Papel}        \\\hline
        Brenddon Gontijo Furtado                                                  & Time de Desenvolvimento \\\hline
        Loana Nunes Valesco                                                       & \textit{Product Owner}    \\\hline
        Fábio Macedo Mendes                                                       & Scrum Master    \\\hline
        \end{tabular}
    \end{table}

    Vale ressaltar que devido ao fato do time de desenvolvimento ser representado por apenas uma pessoa nem todas as práticas do Scrum podem ser aplicadas de maneira efetiva, assim, o \textit{framework} foi acabou sendo utilizado como uma inspiração, buscando principalmente utilizar
    os conceitos de \textit{sprint} e planejamento de \textit{sprint} para auxiliarem nos entregáveis de cada iteração.

    Definiu-se que a duração das \textit{sprints} seriam de 15 dias e ao fim das mesmas seriam realizadas duas reuniões. A primeira, com a \textit{Product Owner}, objetivaria verificar mudanças de escopo, sugestões de melhorias e se o incremento de \textit{software} produzido atendeu as expectativas desejadas. A segunda,
    com o orientador, discutiria decisões arquiteturais, sugestões de implementação e realização
    do planejamento referente à \textit{sprint} posterior.

    Os conceitos e artefatos referentes ao \textit{Backlogs} de produto e \textit{sprint} não foram utilizados de maneira rigorosa, sendo estes registrados de maneira simplificada no arquivo \textit{README}\footnote{\url{https://gitlab.com/brenddongontijo/SME-UnB/blob/master/README.rst}} do repositório hospedeiro do projeto e na ferramenta Gantter \footnote{\url{http://gantter.com/}}.

    \subsection{Teste de \textit{Software}}
    Teste de \textit{software} é o processo de execução de um produto para averiguar se ele atingiu suas especificações e funciona corretamente em seu ambiente alvo \cite{artigo_intro_teste}.

    De acordo com \citeonline{sw_test_tech}, um bom teste é o que possui uma alta probabilidade de encontrar um erro ainda não descoberto e um teste bem sucedido é o que de fato descobre erros desconhecidos.

    Esses testes são estruturados em níveis, cada um com um determinado objetivo dentro do conjunto de testes, de modo a garantir a qualidade do produto em desenvolvimento \cite{sw_test_tech}.

        \subsubsection{Testes Unitários}
        Testes unitários possuem como objetivo verificar a existência de defeitos em cada módulo do projeto. Seu alvo são os métodos desenvolvidos ou pequenos trechos específicos de código \cite{artigo_intro_teste}.

        É realizado durante o desenvolvimento, pelo próprio desenvolvedor, pois testa a unidade básica de \textit{software}, que é o menor ``pedaço''  testável, por sua vez chamado de unidade, dando origem ao nome deste tipo de teste \cite{sw_test_tech}.

        Um exemplo de objetivo do teste unitário é a procura pela identificação de erros de lógica e de implementação \cite{maldonado}.

        \subsubsection{Testes de Integração}
        Possui como objetivo averiguar a existência de falhas relacionadas a interface do \textit{software} entre seus diferentes módulos quando estes são integrados \cite{artigo_intro_teste}.

        É realizado quando uma estrutura maior é formada (devido a integração de dois ou mais módulos), sendo que os módulos possuem suas especificações individuais testadas, porém olhando-se para o conjunto \cite{sw_test_tech}. A medida que essas estruturas vão sendo testadas, a estrutura de programa que foi determinada pelo projeto vai sendo construída \cite{maldonado}.

\section{\textit{Software} Livre}
``O Software livre é aquele que permite aos usuários usá-lo, estudá-lo, modificá-lo e redistribui-lo, em geral, sem restrições para tal e prevenindo que não sejam impostas restrições aos futuros usuários'' \cite{meirelles2013}.

Em comparação ao software restrito, o software livre apresenta algumas vantagens devido ao fato de seu código-fonte estar disponível para qualquer usuário, partindo do princípio que seu licenciamento esteja de acordo com as definições da \textit{Free Software Foundation}\footnote{\url{http://www.gnu.org/philosophy/free-sw.pt-br.html}} ou da \textit{Open Source Initiative}\footnote{\url{https://opensource.org/docs/definition.html}}. Essa disponibilidade auxilia no desenvolvimento de aplicações personalizadas, uma vez que que é possível partir de uma solução já existente ao invés de desenvolver tudo partindo do zero. Tal abordagem possui um impacto significativo na redução de custos e diminuição na duplicação de esforço \cite{meirelles2013}.

Raymond \cite{raymond1999}, observando o modelo de desenvolvimento do Linux, percebeu que o compartilhamento de código possivelmente melhoraria na qualidade final de uma aplicação, uma vez que uma quantidade grande de desenvolvedores, com diferentes habilidades e conhecimentos, conseguem propor melhorias e consertar \textit{bugs} em uma pequena quantidade de tempo \cite{meirelles2013}.

\section{Licença de \textit{software}}
Software é um produto do intelecto humano e consequentemente um tipo de propriedade intelectual, possui um valor de interesse e a lei permite o dono possuí-lo e controlá-lo. Além disso, o dono pode exercer domínio sobre esse software e fornecê-lo, vendê-lo, ou licenciá-lo para que outras pessoas o usem. \cite{rosen_2004}.

A licença utilizada no sistema deste trabalho foi a MIT\footnote{\url{https://opensource.org/licenses/MIT}}. A mesma consiste em uma licença permissiva de software, permitindo o uso comercial, modificação, distribuição e sublicenciamento. Para ser utilizada basta que a mesma esteja incluída junto ao código-fonte \cite{mit_license}.