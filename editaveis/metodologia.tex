\chapter{Metodologia}

\section{Métodos Ágeis}
Os negócios atualmente operam em um ambiente global sujeito a rápidas mudanças sendo crucial
estar preparado para novas oportunidades de mercado, mudanças de condições econômicas e ao surgimento
de produtos e serviços concorrentes. O \textit{software} é um dos componentes cruciais para a realização
de todas as operações de negócio e realizá-lo de forma rápida também é uma maneira de aproveitar novas
oportunidades e responder às pressões competitivas. \cite{sommerville_2006}

O manifesto ágil, \cite{beck2001agile}, consiste na base que fundamenta o desenvolvimento ágil de
\textit{software}, onde o mesmo é composto de quatro valores fundamentais:

\begin{itemize}
    \item Os indivíduos e suas interações acima de procedimentos e ferramentas;
    \item O funcionamento do software acima de documentação abrangente;
    \item A colaboração dos clientes acima da negociação de contratos;
    \item A capacidade de resposta a mudanças acima de um plano pré-estabelecido.
\end{itemize}

Além desses valores são definidos 12 princípios de agilidade:

\begin{itemize}
    \item Nossa maior prioridade é satisfazer o cliente, através da entrega adiantada e contínua de software de valor;
    \item Aceitar mudanças de requisitos, mesmo no fim do desenvolvimento. Processos ágeis se adequam a mudanças, para que o cliente possa tirar vantagens competitivas;
    \item Entregar software funcionando com freqüencia, na escala de semanas até meses, com preferência aos períodos mais curtos;
    \item Pessoas relacionadas à negócios e desenvolvedores devem trabalhar em conjunto e diariamente, durante todo o curso do projeto;
    \item Construir projetos ao redor de indivíduos motivados. Dando a eles o ambiente e suporte necessário, e confiar que farão seu trabalho;
    \item O Método mais eficiente e eficaz de transmitir informações para, e por dentro de um time de desenvolvimento, é através de uma conversa cara a cara;
    \item Software funcional é a medida primária de progresso;
    \item Processos ágeis promovem um ambiente sustentável. Os patrocinadores, desenvolvedores e usuários, devem ser capazes de manter indefinidamente, passos constantes;
    \item Contínua atenção à excelência técnica e bom design, aumenta a agilidade;
    \item Simplicidade: a arte de maximizar a quantidade de trabalho que não precisou ser feito;
    \item As melhores arquiteturas, requisitos e designs emergem de times auto-organizáveis;
    \item Em intervalos regulares, o time reflete em como ficar mais efetivo, então, se ajustam e otimizam seu comportamento de acordo.
\end{itemize}

Nem todos os processos ágeis aplicam tais princípios de maneira igualitária e alguns modelos escolhem ignorar (ou ao menos minimar) a importância de um ou mais princípios. \cite{pressman_2009}

Segundo \cite{sommerville_2006}, processos de desenvolvimento ágeis geralmente são iterativos onde a
especificação, projeto, desenvolvimento e teste são intercalados. O \textit{software} é desenvolvido
em uma série de incrementos e cada incremento fornece uma nova funcionalidade ao sistema. As duas principais
vantagens de se adotar uma abordagem incremental para o desenvolvimento de \textit{software} são:

\begin{itemize}
    \item Entrega acelerada dos serviços ao cliente. Os clientes poderão obter valor do sistema com incrementos iniciais;
    \item Engajamento do usuário com o sistema. Os usuários do sistema devem estar envolvidos no processo
    de desenvolvimento dando \textit{feedback} à equipe de desenvolvimento sobre os incrementos entregues.
\end{itemize}

\section{Teste de \textit{Software}}
Teste de \textit{software} é o processo de execução de um produto para averiguar se ele atingiu suas especificações e funciona corretamente em seu ambiente alvo \cite{artigo_intro_teste}.

De acordo com \citeonline{sw_test_tech}, um bom teste é o que possui uma alta probabilidade de encontrar um erro ainda não descoberto e um teste bem sucedido é o que de fato descobre erros desconhecidos.

Esses testes são estruturados em níveis, cada um com um determinado objetivo dentro do conjunto de testes, de modo a garantir a qualidade do produto em desenvolvimento \cite{sw_test_tech}.

    \subsection{Testes Unitários}
    Testes unitários possuem como objetivo verificar a existência de defeitos em cada módulo do projeto. Seu alvo são os métodos desenvolvidos ou pequenos trechos específicos de código \cite{artigo_intro_teste}.

    É realizado durante o desenvolvimento, pelo próprio desenvolvedor, pois testa a unidade básica de \textit{software}, que é o menor ``pedaço''  testável, por sua vez chamado de unidade, dando origem ao nome deste tipo de teste \cite{sw_test_tech}.

    Um exemplo de objetivo do teste unitário é a procura pela identificação de erros de lógica e de implementação \cite{maldonado}.

    \subsection{Testes de Integração}
    Possui como objetivo averiguar a existência de falhas relacionadas a interface do \textit{software} entre seus diferentes módulos quando estes são integrados \cite{artigo_intro_teste}.

    É realizado quando uma estrutura maior é formada (devido a integração de dois ou mais módulos), sendo que os módulos possuem suas especificações individuais testadas, porém olhando-se para o conjunto \cite{sw_test_tech}. A medida que essas estruturas vão sendo testadas, a estrutura de programa que foi determinada pelo projeto vai sendo construída \cite{maldonado}.

\section{\textit{Extreme Programming}}
Segundo \cite{beck_2004}, \textit{Extreme Programming} (XP) é uma filosofia de desenvolvimento de
\textit{software} baseada nos valores de comunicação, \textit{feedback}, simplicidade, coragem e respeito.
Por ser um tipo de metodologia ágil de desenvolvimento de \textit{software} são defendidas \textit{releases}
frequentes, as quais devem possuir curtos ciclos de desenvolvimento, destinadas à aumentar a produtividade, qualidade de \textit{software} e introduzir pontos de controle para que novas necessidades dos clientes sejam atendidas.

Para \cite{pressman_2009}, o \textit{Feedback} é derivado de três recursos: o \textit{software} implementado, cliente e os outros membros da equipe de \textit{software}. Além disso, outra forma de \textit{feedback} baseia-se nos resultados dos testes de \textit{software}, uma vez que é possível identificar
lacunas que o mesmo possui. O XP utiliza o teste unitário como técnica primária para testes.

Práticas do XP:

\begin{itemize}
    \item Testar cedo, com frequência e automação.
    \item \textit{Design} incremental.
    \item Implantação diária.
    \item Envolvimento do cliente.
    \item Integração contínua.
    \item Ciclos de desenvolvimento curtos.
    \item Planejamento incremental.
    \item Programação em pares.
\end{itemize}

\section{Scrum}
Scrum.
O que é sprint.
Reunião de planejamento de sprint com orientador e cliente.

\section{Software Livre}

\section{Licença de Software}
MIT

\section{Cronograma}
