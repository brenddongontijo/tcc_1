\chapter{Ciclo de Vida do Sistema}
\section{Definição da Abordagem e Metodologia}
Como a parte inicial do projeto, com um período de aproximadamente 6 meses, seria realizada pelo próprio autor, a disponibilidade do cliente ser mais abrangente, o projeto for evoluido conforme o tempo e novas necessidades forem surgindo, definiu-se a utilização de uma abordagem de desenvolvimento ágil \cite{beck2001agile}.


A metodologia utilizada baseou-se no Kanban \cite{radigan_2015} com algumas práticas do \textit{Extreme Programming}.

A metodologia definida foi o  \cite{beck_2004}, visando sempre estar evoluindo o sistema com \textit{feedbacks} do cliente, produzindo código com padrões de codificação e simples para possíveis manuntenções no futuro.



Matheus Faria. Canbam. Produto nao finalizado, crescendo aos poucos, Sutherland. ciclo de desenvolvimento sem ter fim, pois projeto é grande.

Para auxiliar nas \textit{releases} do projeto, utilizou-se o conceito de \textit{sprints}, provinda do \textit{Scrum} \cite{scrum_guide}. As \textit{sprints} são pequenos intervalos de tempo, de um mês ou menos, durante o qual uma versão incremental potencialmente utilizável do produto é criada. Uma nova Sprint inicia imediatamente após a conclusão da Sprint anterior.

\section{Contexto e Necessidades}
Tendo como problema a falta de monitoramento energético adequado na Universidade de Brasília, foi realizado um estudo sobre o contexto para identificar as necessidades do cliente. Tal estudo abordava o entendimento de fatores energéticos cruciais, como por exemplo sistemas trifásicos, horários de ponta e fora de ponta, tensão, corrente, resistência e afins.

Explicar as grandezas, energia custa dinheiro, etc. redirecionar um gasto em ponta para fora de ponta.

Com um entendimento melhor do contexto, algumas reuniões foram feitas com o cliente e foram obtidas as seguintes necessidades:

\begin{itemize}
    \item Cada campus da Universidade de Brasília deve ser responsável por realizar seu próprio monitoramento energético.
    \item O campus Darcy Ribeiro deve funcionar como uma espécie de administração central.
    \item As medições devem ser registradas e poderão ser resgatadas, conforme um período de tempo especificado.
\end{itemize}

Cliente loana e alex representando unb, como cliente.

Buscando atender toda a conectividade esperada pelo projeto e trazer benefícios ao cliente, definiu-se que sistema teria uma implementação \textit{web}. Sistemas \textit{web} não necessitam de instalação por parte do usuário final, facilitando bastante o processo de implantação e podem ser acessados a partir de qualquer lugar e em qualquer plataforma \cite{pressman_2009}.

Um dos pontos levantados nas reuniões foi o de um possível monitoramento, no futuro, de medições de água. Tendo em vista esses recursos que seriam monitorados e de possíveis outros, definiu-se o nome do projeto como Sistema de Monitoramento de Insumos - Universidade de Brasília (SMI-UnB). Definiu-se, também, que para o contexto inicial seriam enfatizados os dados de energia e que seriam entregues duas \textit{releases}, de aproximadamente 80 dias, sendo essas:

\begin{itemize}
    \item Release 1: coleta dos dados de energia.
    \item Release 2: apresentação dos dados e prorótipo de comunicação inter-campi.
\end{itemize}

Deginiu-se que a hospedagem do código/documentação do projeto e o controle de mudanças seriam realizados utilizando-se o software livre GitLab CE\footnote{\url{https://gitlab.com/gitlab-org/gitlab-ce}} e para controlamento de versão, a ferramenta Git\footnote{\url{https://git-scm.com/}}.

A escolha do GitLab CE se deve pela cultura de software livre, pois o projeto teria um âmbito acadêmico, o compartilhamento de conhecimento poderia ser realizado de maneira mais fácil e a qualidade final não seria comprometida \cite{raymond1999}.

A licença escolhida para o projeto foi a MIT. Ela consiste em uma licença permissiva, autorizando o uso comercial, modificação, distribuição e sublicenciamento \cite{mit_license}

Grande uso meio academico, desenvolvida por uma universidade importante. A unb nao tem um modelo de refêrencia para software livre. o sublicenciamento seria necessário para a unb poder mudar a licença depois.

\section{Requisitos}
Com as necessidades em mãos e o repositório do projeto criado, utilizou-se o propósito das estórias de usuário \cite{beck_2004}, de uma maneira mais simplificada, em \textit{milestones} no repositório, as quais possuiriam diversas \textit{issues} e essas \textit{issues} apresentariam uma terminologia mais técnica da solução em si. Uma \textit{milestone} pode possuir uma data início/fim e é composta por um aglomerado de problemas, comumente chamados de \textit{issues} \cite{gitlab}.

Vale ressaltar que as \textit{milestones} e \textit{issues} foram criadas acompanhando o decorrer do projeto, ou seja, a cada \textit{sprint}, era avaliado se ocorreram mudanças de escopo, quais seriam as \textit{issues} da próxima \textit{sprint}, se era necessário criar uma nova \textit{milestone} etc.

\section{Implantação - Devops}
Contínuo, desenvolvimento com implantação.

Filosofia, nível de automatização alto.

Tasks

Containers utilizados.

\section{Desenvolvimento}
Será explicado depois