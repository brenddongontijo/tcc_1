\chapter{Visão Geral do Sistema Desenvolvido}

O levantamento adequado de requisitos de \textit{software} é crucial para
que possa ser feito um mapeamento das reais necessidades do cliente com
funcionalidades que um sistema deve atender. Sommerville \cite{sommerville_2006}
destingue bem essas abordagens como ``requisitos do usuário'' e ``requisitos do sistema''. Os requisitos de usuários consistem de declarações, em
linguagem natural, que o sistema deve fornecer e restrições que o mesmo deve
operar. Os requisitos de sistema estabelecem de maneira detalhada as
funções e restrições do sistema. Além disso, os requisitos de sistema
classificam-se em funcionais e não funcionais:

\begin{itemize}
    \item Requisitos Funcionais: declaração de funções que o sistema deve fornecer, como o mesmo irá reagir com certas entradas e como deve se comportar para certas situaçoes.
    \item Requisitos Não Funcionais: restrições sobre serviços ou funções oferecidas pelo sistema.
\end{itemize}

Objetivando levantar os requisitos do sistema, realizaram-se reuniões com o
cliente e os requisitos obtidos foram os seguintes:

\begin{itemize}
    \item Funcionais:
    \begin{itemize}
        \item O sistema deve ser capaz de realizar um monitoramento temporal de recursos energéticos.
        \item O sistema deve ser capaz de gerar gráficos com as medições de energia obtidas.
        \item O sistema deve permitir a autenticação de usuários com diferentes níveis de acesso.
        \item O sistema deve permitir o gerenciamento de usuários, prédios e aparelhos de medição.
        \item O sistema deve permitir que usuários atualizem suas informações básicas.
    \end{itemize}
    \item Não Funcionais:
    \begin{itemize}
        \item O
    \end{itemize}
\end{itemize}

Com os requisitos obtidos, realizou-se uma atribuição dos mesmos para
as iterações do projeto, onde cada iteração possuiria um ou mais requisitos,
que eram mapedos, de maneira mais simplificada, em uma \textit{milestone} no repositório oficial do projeto.
As \textit{milestone} representariam um requisito e possuiriam um conjunto
de tarefas (\textit{issues}), que após serem finalizadas, concluiriam
o requisito representado pela mesma.