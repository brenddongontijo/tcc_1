\begin{resumo}[Abstract]
 \begin{otherlanguage*}{english}
   Energy consumption has been growing over the years, which implies the need to create new infrastructures with diverse social, environmental and economic impacts. The rational use of energy is able to reduce these impacts, providing a more sustainable development. In view of these issues, the University of Brasilia has created an energy monitoring initiative for its campus with the design of a system for such purposes. The objective of this work is to develop this monitoring system and, from that point on, to promote more rational energy use policies within the University.

   The development of the system will be based on the requirements established with Campus City Hall and will use the knowledge and methods of \textit{Software} Engineering, applying them in the system life cycle. The present work performs the collection and presentation of energy data, but can be extended to other types of inputs. The concepts used address agile development practices, free \textit{software}, \textit{software} configuration management, and web technologies. This paper presents how the system life cycle and all its important characteristics were detailed.

   \vspace{\onelineskip}

   \noindent
   \textbf{Key-words}: \textit{Software} Engineering. \textit{Software} Configuration Management. Agile Methods. Free \textit{Software}. Energy Monitoring.
 \end{otherlanguage*}
\end{resumo}
