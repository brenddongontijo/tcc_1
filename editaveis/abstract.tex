\begin{resumo}[Abstract]
 \begin{otherlanguage*}{english}
   Irrational energetic consumption has been growing over the years. Public agencies should set the example to the population, having a more conscious consumption, however, many Brazilian agencies still do not have any project that aims at their energy efficiencies. The University of Brasilia, undergoing this problem, decided to make an investment for these issues, aiming at an adequate energy monitoring for its campus. With this monitoring being carried out effectively, it would be possible to carry out possible policies to reduce energy consumption. The objective of this work is to use the knowledge and methods of Software Engineering, applying them during the development of a web system for monitoring inputs, initially aimed at energy data, from the University of Brasília. The concepts used address agile development practices, free software, software configuration management and web technologies. This paper presents how the system life cycle and all its important characteristics were detailed.

   \vspace{\onelineskip}

   \noindent
   \textbf{Key-words}: Software Engineering. Software Configuration Management. Agile Methods. Free Software. Energy Monitoring.
 \end{otherlanguage*}
\end{resumo}
