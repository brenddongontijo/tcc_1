\chapter{Requisitos do Sistema}
Uma das técnicas comumente utilizadas para a elicitação de requisitos é a entrevista com questionários \cite{goguen_1994}.

Objetivando levantar os requisitos do sistema, realizaram-se reuniões com o
cliente e os requisitos obtidos foram os seguintes:

\begin{itemize}
    \item Funcionais:
    \begin{itemize}
        \item O sistema deve ser capaz de realizar um monitoramento temporal de recursos energéticos.
        \item O sistema deve ser capaz de gerar gráficos com as medições de energia obtidas.
        \item O sistema deve permitir a autenticação de usuários com diferentes níveis de acesso.
        \item O sistema deve permitir o gerenciamento de usuários, prédios e aparelhos de medição.
        \item O sistema deve permitir que usuários atualizem suas informações básicas.
    \end{itemize}
    \item Não Funcionais:
    \begin{itemize}
        \item O
    \end{itemize}
\end{itemize}

Com os requisitos obtidos, realizou-se uma atribuição dos mesmos para
as iterações do projeto, onde cada iteração possuiria um ou mais requisitos,
que eram mapedos, de maneira mais simplificada, em uma \textit{milestone} no repositório oficial do projeto.
As \textit{milestone} representariam um requisito e possuiriam um conjunto
de tarefas (\textit{issues}), que após serem finalizadas, concluiriam
o requisito representado pela mesma.