\begin{resumo}
O consumo enérgico irracional vem crescendo com o passar dos anos. Os órgãos públicos deveriam dar o exemplo à população, tendo um consumo mais consciente, porém, muitos órgãos brasileiros ainda não possuem nenhum projeto que visa suas eficiências energéticas. A Universidade de Brasília, passando por esse problema, decidiu realizar um investimento para essas questões, objetivando um monitoramento energético adequado para seus câmpus. Com esse monitoramento sendo realizado de maneira efetiva, seria possível a realização de possíveis políticas para a redução do consumo de energia. O objetivo deste trabalho é utilizar os conhecimentos e métodos da Engenharia de Software, aplicando-os durante o desenvolvimento de um sistema \textit{web} para monitoramento de insumos, visando inicialmente os dados energéticos, da Universidade de Brasília. Os conceitos utilizados abordam práticas de desenvolvimento ágeis, software livre, gerência de configuração de software e tecnologias \textit{web}. Este trabalho apresenta como foi o ciclo de vida do sistema e detalhadamente todas as suas características importantes.

 \vspace{\onelineskip}

 \noindent
 \textbf{Palavras-chaves}: Engenharia de Software. Gerência de Configuração de Software. Desenvolvimento Ágil de Software. Software Livre. Monitoramento Energético.
\end{resumo}