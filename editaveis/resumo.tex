\begin{resumo}
O consumo energético vem crescendo com o passar dos anos, o que implica na necessidade de criação de novas infraestruturas com diversos impactos sociais, ambientais e econômicos. O uso racional de energia é capaz de diminuir esses impactos, proporcionando um desenvolvimento mais sustentável. Tendo em vista essas questões, a Universidade de Brasília criou uma iniciativa de monitoramento energético para seus \textit{campi} com a idealização de um sistema para tais fins. O objetivo deste trabalho é desenvolver esse sistema de monitoramento e, a partir desse ponto, fomentar políticas de uso mais racional de energia dentro da Universidade.

O desenvolvimento do sistema será baseado nos requisitos definidos junto à Prefeitura de Campus e utilizará os conhecimentos e métodos da Engenharia de \textit{Software}, aplicando-os no ciclo de vida do sistema. O presente trabalho realiza a coleta e apresentação dos dados energéticos, mas pode ser estendido para outros tipos de insumos. Os conceitos utilizados abordam práticas de desenvolvimento ágeis, \textit{software} livre, gerência de configuração de \textit{software} e tecnologias \textit{web}. Este trabalho apresenta como foi o ciclo de vida do sistema e detalhadamente todas as suas características importantes.

 \vspace{\onelineskip}

 \noindent
 \textbf{Palavras-chaves}: Engenharia de \textit{Software}. Gerência de Configuração de \textit{Software}. Desenvolvimento Ágil de \textit{Software}. \textit{Software} Livre. Monitoramento Energético.
\end{resumo}