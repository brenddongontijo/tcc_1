\begin{resumo}
 O objetivo deste trabalho é utilizar os conhecimentos e métodos da Engenharia de Software aplicando-os durante o desenvolvimento de um sistema \textit{web} para monitoramento energético da Universidade de Brasília. Os conceitos utilizados abordam práticas de desenvolvimento ágeis, software livre, gerência de configuração de software e tecnologias \textit{web}. Este trabalho apresenta a evolução arquitetural e as decisões tomadas durante as iterações do projeto.

 \vspace{\onelineskip}
    
 \noindent
 \textbf{Palavras-chaves}: Engenharia de Software. Gerência de Configuração de Software. Desenvolvimento Ágil de Software. Software Livre. Monitoramento Energético.
\end{resumo}
