\chapter{Introdução}
O consumo de energia elétrica vem aumentando com o passar dos anos. No ano de 2014, o Brasil consumiu 531.1 TWh, o que corresponde a um aumento de 2.9\% comparado a 2013. Cerca de 2.4\% deste total é consumido por prédios públicos \cite{balanco_energetico}. No ano de 2015, foi realizado um reajuste tarifário da energia elétrica correspondente a 33\% para residências e 32,5\% para empresas, indústrias e comércios \cite{aumento_energia}. Tendo em vista esse cenário, é interessante a administração pública viabilizar políticas para o uso mais racional da energia elétrica.

Em Maio de 2016, a Prefeitura de Campus da Universidade de Brasília (UnB) aprovou um projeto de monitoramento energético da própria Universidade, objetivando que cada campus da instituição (Asa Norte, Ceilândia, Gama e Planaltina) seja responsável por realizar seu próprio monitoramento de energia e enviar essas informações para a administração central.

O presente trabalho consiste no desenvolvimento de um sistema web capaz de unificar, de forma setorial, os insumos utilizados pelos \textit{campi} da UnB, com o intuito de reduzir gastos e promover a sustentabilidade. Inicialmente foi realizado a coleta de grandezas energéticas, porém, o projeto possui a possibilidade de extensão para outros insumos. Os monitoramentos realizados foram efetivados por meio de equipamentos eletrônicos denominados transdutores, que foram instalados nos quadros de energia para transmitir suas informações para o sistema através da rede de um campus.

A apresentação das grandezas energéticas monitoradas por um transdutor foi realizada por meio de um gráfico de linhas, tornando possível que usuários do sistema consigam visualizar de maneira simples e fácil as medições de um período de tempo específico.

O desenvolvimento do sistema teve como base fundamentos e metodologias presentes na Engenharia de \textit{Software}, com o objetivo de se obter um \textit{software} confiável e eficiente de maneira econômica e sustentável.

Procedimentos denominados métodos ágeis guiaram o ciclo de vida do sistema de maneira iterativa e incremental, procurando trazer \textit{software} funcional e com qualidade em curtos períodos de tempo. Atrelado a essa abordagem, adotou-se boas práticas de desenvolvimento, referentes ao Kanban e \textit{Extreme Programming}. Essas práticas se adequavam ao contexto vigente e auxiliaram na entrega de um produto que atendeu as expectativas do cliente.

Foi necessário utilizar os conhecimentos da Gerência de Configuração de \textit{Software} para que o sistema respondesse às volatilidades presentes no decorrer do desenvolvimento de \textit{software}, garantindo, assim, sua manutenibilidade, confiabilidade e qualidade.

Algumas filosofias do processo de desenvolvimento de \textit{software} Devops foram utilizadas no decorrer do projeto, buscando facilitar sua automatização e distribuição.

\section{Objetivo Geral}
Desenvolver parte inicial de um sistema \textit{web} capaz de monitorar temporalmente insumos da Universidade de Brasília, com auxílio de conhecimentos e métodos da Engenharia de \textit{Software}.

\subsection{Objetivos Específicos}

\begin{itemize}
    \item Coleta, armazenamento e análise de informações energéticas;
    \item API de comunicação com sistemas externos;
    \item Realizar protótipo para comunicação entre uma administração central e os \textit{campi} da UnB.
\end{itemize}

\section{Metodologia Utilizada}
A metodologia abordada neste trabalho teve como base o desenvolvimento de \textit{software} livre, utilizando-se os princípios empíricos de desenvolvimento ágil de \textit{software} juntamente com o Kanban e algumas práticas do \textit{Extreme Programming}.

\section{Estruturação do Trabalho}
O trabalho se encontra estruturado da seguinte maneira:

\begin{itemize}
    \item Capítulo 2: métodos empíricos da Engenharia de \textit{Software} utilizados para
    auxiliar no desenvolvimento do sistema.
    \item Capítulo 3: ciclo de vida.
    \item Capítulo 4: características e tecnologias adotadas no sistema.
    \item Capitulo 5: conclusão e trabalhos futuros.
\end{itemize}