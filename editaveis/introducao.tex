\chapter{Introdução}
O crescimento no consumo de energia elétrica vem aumentando com o passar dos anos. No
ano de 2014, o Brasil consumiu 531.1 TWh, o que corresponde a um aumento de 2.9\% comparado à 2013. Cerca de 2.4\% deste total é consumido por prédios públicos \cite{balanco_energetico}. No ano de 2015, foi realizado um reajuste tarifário da energia elétrica correspondente à 33\% para clientes residenciais e 32,5\% para empresas, indústrias e comércios \cite{aumento_energia}. Tendo em vista este cenário, é interessante a administração pública viabilizar políticas para o uso mais racional da energia elétrica.

Em Maio de 2016, a Prefeitura de Campus da Universidade de Brasília (UnB) aprovou um projeto de gerenciamento energético da própria Universidade, objetivando que cada Campus da instituição (Asa Norte, Ceilândia, Gama e Planaltina) seja responsável por realizar seu próprio monitoramento de energia e enviar essas informações coletadas para a administração central. O objetivo do presente trabalho é desenvolver um Sistema de Monitoramento Energético (SME-UnB) utilizando tecnologias \textit{web} e disponibilizado sobre uma licença livre. Tal sistema criaria subsídios para a administração gerenciar a energia utilizada com o intuito de reduzir gastos e promover a sustentabilidade.

O desenvolvimento do sistema deve se basear nos princípios estabelecidos pela Engenharia de Software para obter, de maneira econômica e sustentável, software confiável e eficiente \cite{naur_1969}. Em uma definição mais formal, Engenharia de Software baseia-se na ``aplicação de uma abordagem sistemática, disciplinada e quantificável para o desenvolvimento, operação e manutenção de software'' \cite{ieee_glossary}.

Este trabalho inspira-se na filosofia e em procedimentos dos denominados métodos ágeis \cite{beck2001agile} e entende que estas metodologias seriam as mais eficazes para a solução de software proposta. Dentre as práticas de desenvolvimento adotadas utilizou-se alguns princípios empíricos de \textit{Extreme Programming} \cite{beck_2004} e \textit{Scrum} \cite{scrum_guide}.

Tendo em vista o curto período para desenvolvimento e requisitos inicialmente difusos, foi necessário utilizar boas práticas de Gerência de Configuração de Software \cite{SWEBOK2014} para garantir a manutenibilidade, confiabilidade e qualidade do sistema.

Finalmente, adotou-se uma implementação \textit{web}, pois esses sistemas oferecem diversos benefícios ao cliente \cite{pressman_2009}. O fato dos equipamentos eletrônicos, responsáveis por realizar a leitura dos dados de energia, estarem conectados na rede da Universidade pressupõem conectividade. Sistemas \textit{web} não necessitam de instalação por parte do usuário final facilitando bastante o processo de implantação e podem ser acessados a partir de qualquer lugar e em qualquer plataforma. Com o passar dos anos, sistemas web evoluíram em sofisticadas ferramentas de computação, as quais fornecem não apenas funções autônomas ao usuário final, mas também integram-se com bancos de dados corporativos e aplicativos de negócios.

\section{Objetivo Geral}
Desenvolver sistema \textit{web} capaz de monitorar, em tempo real, medições de energia coletadas por transdutores\footnote{Dispositivo capaz de converter um tipo de energia de entrada em outro de saída.}
instalados em quadros de energia da Universidade de Brasília.

\subsection{Objetivos Específicos}
O sistema deve ser capaz de:
\begin{itemize}
    \item Cadastrar/Editar/Remover transdutores.
    \item Utilizar rede da Universidade para comunicar-se com transdutores.
    \item Realizar a coleta de dados de forma automatizada e temporizada.
    \item Mostrar aos usuários os dados de energia coletados.
\end{itemize}

\section{Metodologia Utilizada}
A metodologia abordada neste trabalho teve como base o desenvolvimento de um \textit{software} livre \footnote{\url{http://www.gnu.org/philosophy/free-sw.pt-br.html}} utilizando-se dos princípios empíricos de desenvolvimento ágil de \textit{software}\footnote{\url{http://agilemanifesto.org/}} juntamente com algumas práticas do Scrum \footnote{\url{http://www.scrumguides.org/}} e \textit{Extreme Programming} \footnote{\url{http://www.extremeprogramming.org/}}.

\section{Estruturação do Trabalho}
O trabalho encontra-se estruturado da seguinte maneira:

\begin{itemize}
    \item Capítulo 2: aborda os métodos empíricos em engenharia de \textit{software} utilizados para
    auxiliar no desenvolvimento do sistema.
    \item Capítulo 3: aborda como estruturou-se a gerência de configuração.
    \item Capítulo 4: aborda as principais tecnologias \textit{web} e protocolos de comunicação utilizados.
    \item Capítulo 5: aborda a evolução do desenvolvimento do sistema, relatando o que foi realizado durante cada \textit{sprint}.
    \item Capitulo 6: aborda possíveis trabalhos futuros.
\end{itemize}