\chapter{Introdução}
A criação de novos produtos que facilitam ou substituem o trabalho manual cada vez mais vem auxiliando na vida dos seres humanos. A Engenharia de Produto surgiu para conduzir a inovação de produtos partindo de sua ideia inicial até todo seu ciclo de vida (\textit{design}, desenvolvimento, fabricação, uso e reciclagem) \cite{ortloff2014}.

Dentre os diferentes produtos existentes encontra-se o software, a IEEE \cite{ieee_glossary} o define como um ``conjunto de programas de computador, procedimentos e possível documentação, além de dados associados ao funcionamento de um sistema de computador''. Uma das diferenças do software, comparado a outros produtos, se dá pelo fato do mesmo ser um conhecimento incorporado, sendo este inicialmente disperso, tácito, latente e incompleto \cite{baetjer1997}. Para Sommerville \cite{sommerville_2006}, software, por ser abstrato e intangível, não pode ser limitado por materiais ou controlado por leis da física ou por processos de manufatura.

A Engenharia de Software consiste no estabelecimento e uso de princípios de engenharia para obter-se, de maneira econômica, software confiável e eficiente \cite{naur_1969}. Em uma definição mais formal, Engenharia de Software baseia-se na ``aplicação de uma abordagem sistemática, disciplinada e quantificável para o desenvolvimento, operação e manutenção de software; isto é, a aplicação da engenharia ao software'' \cite{ieee_glossary}.

Dentre as áreas da Engenharia de Software existe a Gerência de Configuração de Software, responsável por auxiliar o desenvolvimento provendo controle de versão, mudança e auditoria de configurações \cite{SWEBOK2014}. Roger Pressman \cite{pressman_2009} a define como um ``conjunto de atividades projetadas para controlar as mudanças pela identificação dos produtos do trabalho que serão alterados, estabelecendo um relacionamento entre eles, definindo o mecanismo para o gerenciamento de diferentes versões destes produtos, controlando as mudanças impostas, e auditando e relatando as mudanças realizadas''.

Visando juntar a capacidade computacional com conteúdo informativo foram criados os sistemas e aplicativos baseados na \textit{web}. Com o passar dos anos, esses sistemas evoluíram em sofisticadas ferramentas de computação, as quais não forneciam apenas funções autônomas ao usuário final, mas também integrariam-se com bancos de dados corporativos e aplicativos de negócios \cite{pressman_2009}.

A proposta deste trabalho consiste na aplicação de conceitos, técnicas e ferramentas de Engenharia de Software em um contexto de energia.

\section{Objetivo Geral}
Desenvolver sistema \textit{web} capaz de monitorar, em tempo real, medições de energia coletadas por transdutores\footnote{Dispositivo capaz de converter um tipo de energia de entrada em outro de saída.}
instalados em quadros de energia da Universidade de Brasília.

\subsection{Objetivos Específicos}
O sistema deve ser capaz de:
\begin{itemize}
    \item Cadastrar/Editar/Remover transdutores.
    \item Utilizar rede da Universidade para comunicar-se com transdutores.
    \item Realizar a coleta de dados de forma automatizada e temporizada.
    \item Mostrar aos usuários os dados de energia coletados.
\end{itemize}

\section{Metodologia Utilizada}
A metodologia abordada neste trabalho teve como base o desenvolvimento de um \textit{software} livre \footnote{\url{http://www.gnu.org/philosophy/free-sw.pt-br.html}} utilizando-se dos princípios empíricos de desenvolvimento ágil de \textit{software}\footnote{\url{http://agilemanifesto.org/}} juntamente com algumas práticas do Scrum \footnote{\url{http://www.scrumguides.org/}} e \textit{Extreme Programming} \footnote{\url{http://www.extremeprogramming.org/}}.

\section{Estruturação do Trabalho}
O trabalho encontra-se estruturado da seguinte maneira:

\begin{itemize}
    \item Capítulo 2: abordará os métodos empíricos em engenharia de \textit{software} utilizados para
    auxiliar no desenvolvimento do sistema.
    \item Capítulo 3: abordará como estruturou-se a gerência de configuração.
    \item Capítulo 4: abordará as princípais tecnologias \textit{web} e protocolos de comunicação utilizados.
    \item Capítulo 5: abordará a evolução do desenvolvimento do sistema, relatando o que foi realizado durante cada \textit{sprint}.
    \item Capitulo 6: abordará possíveis trabalhos futuros.
\end{itemize}