\chapter{Introdução}
O crescimento no consumo de energia elétrica vem aumentando com o passar dos anos. No
ano de 2014, o Brasil consumiu 531.1 TWh, o que corresponde a um aumento de 2.9\% comparado a 2013. Cerca de 2.4\% deste total é consumido por prédios públicos \cite{balanco_energetico}. No ano de 2015, foi realizado um reajuste tarifário da energia elétrica correspondente a 33\% para residências e 32,5\% para empresas, indústrias e comércios \cite{aumento_energia}. Tendo em vista esse cenário, é interessante a administração pública viabilizar políticas para o uso mais racional da energia elétrica.

Em Maio de 2016, a Prefeitura de Campus da Universidade de Brasília (UnB) aprovou um projeto de monitoramento energético da própria Universidade, objetivando que cada campus da instituição (Asa Norte, Ceilândia, Gama e Planaltina) seja responsável por realizar seu próprio monitoramento de energia e enviar essas informações para a administração central. Um sistema \textit{web} poderia tornar possível o monitoramento temporal dos insumos da Universidade, provendo subsídios para a administração ter conhecimento de como a energia está sendo utilizada, com o intuito de reduzir gastos e promover a sustentabilidade.

A Engenharia de Software baseia-se na ``aplicação de uma abordagem sistemática, disciplinada e quantificável para o desenvolvimento, operação e manutenção de software'' \cite{ieee_glossary}, sendo capaz de providenciar, de maneira econômica e sustentável, software confiável e eficiente \cite{naur_1969}.

Procedimentos denominados métodos ágeis \cite{beck2001agile} guiam o ciclo de vida de um software de maneira iterativa e incremental, procurando sempre trazer software funcional e com qualidade em curtos períodos de tempo. Atrelado a essa abordagem, adotar boas práticas de desenvolvimento, como o Kanban \cite{radigan_2015} e \textit{Extreme Programming} \cite{beck_2004}, direciona as equipes auto-gerenciáveis, de maneira mais eficiente, a produzirem software que atenda as expectativas de seus clientes.

O processo de desenvolvimento de software \textit{Devops} \cite{loukides_2012}, em conjunto com a Gerência de Configuração de Software \cite{SWEBOK2014}, busca automatizar e distribuir um sistema, garantindo que esse seja manutenível, confiável e tenha qualidade.

\section{Objetivo Geral}
Desenvolver parte inicial de um sistema \textit{web} capaz de monitorar temporalmente insumos da Universidade de Brasília, com auxílio de conhecimentos e métodos da Engenharia de Software.

\subsection{Objetivos Específicos}

\begin{itemize}
    \item Tornar possível a coleta de informações energéticas;
    \item Realizar protótipo para comunicação entre uma administração central e os câmpus da UnB.
\end{itemize}

\section{Metodologia Utilizada}
A metodologia abordada neste trabalho teve como base o desenvolvimento de um software livre, utilizando-se os princípios empíricos de desenvolvimento ágil de software juntamente com o Kanban e algumas práticas do \textit{Extreme Programming}.

\section{Estruturação do Trabalho}
O trabalho se encontra estruturado da seguinte maneira:

\begin{itemize}
    \item Capítulo 2: aborda os métodos empíricos da Engenharia de Software utilizados para
    auxiliar no desenvolvimento do sistema.
    \item Capítulo 3: aborda como foi o ciclo de vida do sistema.
    \item Capítulo 4: aborda as característias do sistema.
    \item Capitulo 5: aborda possíveis trabalhos futuros.
\end{itemize}